\documentclass[12pt]{article}
\usepackage[russian]{babel}
\usepackage{indentfirst}
\usepackage{graphicx}
\usepackage{mathtools}
\usepackage [left=30 mm, top=15 mm, right=30 mm, bottom=20mm, nohead, footskip=10 mm] {geometry}

\parindent=24pt

\begin{document}
	\thispagestyle{empty}

\begin{center}
	\large{МИНОБРНАУКИ РОССИИ} \par
	\vspace{0.3cm}
	\normalsize
	{ФЕДЕРАЛЬНОЕ ГОСУДАРСТВЕННОЕ АВТОНОМНОЕ ОБРАЗОВАТЕЛЬНОЕ УЧРЕЖДЕНИЕ ВЫСШЕГО ОБРАЗОВАНИЯ} \par
	\vspace{0.3cm}
	\textbf{\guillemotleft САНКТ-ПЕТЕРБУРГСКИЙ ПОЛИТЕХНИЧЕСКИЙ}
	\textbf{УНИВЕРСИТЕТ ПЕТРА ВЕЛИКОГО\guillemotright} \par
	\vspace{0.3cm}
	
	{Институт компьютерных наук и кибербезопасности}\par
	{Высшая школа технологий искусственного интеллекта}\par
	{Направление 02.03.01 Математика и Компьютерные науки}	
\end{center}

\vfill

\begin{center}
	{\LARGE КУРСОВАЯ РАБОТА} \par
	\vspace{0.3cm}
	{\large по дисциплине \guillemotleft Параллельное программирование на суперкомпьютерных системах\guillemotright}\par
	{\LARGE Параллельное программирование на СК: обращение матриц\\ }\par
\end{center}

\vfill

\begin{flushleft}
	Студент: \hspace{1.8cm} \rule[0pt]{2.5cm}{0.5pt}\hfill Жилкина Лада Михайловна\par
	\vspace{1.5cm}
	Преподаватель: \hspace{0.55cm} \rule[0pt]{2.5cm}{0.5pt}\hfill  Лукашин Алексей Андреевич
\end{flushleft}

\vspace{0.5cm}

\begin{flushright}
	\guillemotleft \rule[0pt]{0.8cm}{0.5pt}\guillemotright \rule[0pt]{2cm}{0.5pt} 20\rule[0pt]{0.5cm}{0.5pt} г.
\end{flushright}

\vfill

\begin{center}
	Санкт-Петербург -- 2025
\end{center}
\thispagestyle{empty}
	
	
	\newpage
	\tableofcontents
	
	
	\newpage
	\section{Алгоритмы решения задачи и подходы к распараллеливанию кода}
	\subsection{Определение обратной матрицы}
	
	\par Обратная матрица - такая матрица \(A^{-1}\), при умножении которой на исходную матрицу \(A\) получается единичная матрица \(I\): \[AA^{-1}=A^{-1}A=I\]
	\par Матрица обратима тогда и только тогда, когда она невырождена, то есть её определитель не равен нулю. Для неквадратных матриц и вырожденных матриц обратных матриц не существует.
	
	\subsection{Способы нахождения обратных матриц}
	\paragraph{Метод Жордана--Гаусса.}
	Строится расширенная матрица \([A|I]\), матрица \(A\) последовательно приводится к единичной преобразованием строк (или столбцов). Метод характеризуется высокой последовательностью вычислений, так как каждый шаг зависит от предыдущего. Внутри шага мажно распараллелить операции над строками (умножение, вычитание) и нормализацию строк.
	
	\paragraph{Метод LU--разложения.} Исходная матрица A представляется в виде \(A=LU\), где \(L\) -- нижнетреугольная \(U\) -- верхнетреугольная. Тогда обратную матрицу \(A^{-1}=U^{-1}L^{-1}\) можно найти решением системы \[AX=I \implies L(UX)=I\]
	\par Разложение можно распараллелить по блокам. После разложения обращение сводится к решению нескольких систем с правыми частями -- столбцами единичной матрицы. Решение для каждого столбца \(I\) независимо, что даёт высокую степень параллелизма.
	
	
	\paragraph{Метод QR--разложения.} Исходная матрица \(A\) представляется в виде \(A=QR\), где \(Q\) -- ортогональная,\(R\) -- верхнетреугольная. Обратная матрица вычисляется как \(A^{-1}=R^{-1}Q^T\).
	\par Использование ортогональных преобразований минимизирует накопление ошибок округления и повышает устойчивость при обработке плохо обусловленных систем. Требует большего объёма операций по сравнению с LU-разложением, но хорошо подходит для распараллеливания, так как ключевые этапы могут выполняться независимо для различных подблоков матрицы.
	
	\paragraph{Итерационный метод Ньютона-Шульца.} Находится приближение \(X \approx A^{-1}\) итерационно: \[X_{k+1}=X_k(2I-AX_k)\] 
	\par Требуется начальное приближение \(X_0\), которое можно выбрать как \[X_0=\frac{A^T}{||A||_0||A||_\infty}\]
	\par Основные операции в итерационном процессе -- матричное умножение и вычитание, что позволяет эффективно распараллеливать вычисления. Может демонстрировать нестабильность или медленную сходимость для плохо обусловленных матриц. Может быть избыточным для матриц небольших размеров из-за накладных расходов на организацию итераций и параллельных вычислений.
	\par Из всех рассмотренных методов метод Ньютона-Шульца представляет наибольший интерес с точки зрения распараллеливания и будет реализован в ходе работы.
	
\end{document}